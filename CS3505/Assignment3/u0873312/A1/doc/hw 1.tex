\documentstyle[11pt]{article}
%\usepackage{listings}

\title{A1 : C++ and Shell Warm-up}
\author{Nick(Mykola) Pershyn}

\begin{document}
\maketitle{}
a) This is the command I would use to make my program "\texttt{inch\_to\_cm}" convert numbers written in file "\texttt{in}" and write them to file "\texttt{out}":

\begin{verbatim}
./inch_to_cm < ./in > ./out
\end{verbatim}

b) This is the output:
\begin{verbatim}
0.299999999999999988897769753748
\end{verbatim}

In binary this number would require infinite amount of digits to represent, so when we enter \texttt{0.3} computer get the closest binary number. If I multiply that number by some power of two I would get an integer number.
So "\texttt{0.3*256*256*256*256*256*256*64}" is exactly:
\begin{verbatim}
5404319552844595.000000000000000000000000000000
\end{verbatim}

\end{document} 
